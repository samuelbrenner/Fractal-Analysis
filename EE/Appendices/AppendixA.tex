% Appendix A
\chapter{Source Code of Computational Modules Used} % Main appendix title
\usepackage{listings}
\lstset{language=C++} 
\label{AppendixA} % For referencing this appendix elsewhere, use \ref{AppendixA}

\lhead{Appendix A. \emph{Source Code}} % This is for the header on each page - perhaps a shortened title

\begin{lstlisting}
/** Fractal analysis module utilizing boxcounting to determine 
	the fractal dimension of a shape in the input text/binary file.

	Prints to terminal:
		- the results at each level of analysis
		- the overall dimension as determined by least-squares regression
		- an R^2 value
		
	@author Samuel Brenner
	@version July 11, 2013

**/



#include <fstream>
#include <string>
#include <sstream>
#include <math.h>
#include <iostream>
#include "regressionModule.h"
#include "dataReader.h"
#include "interpolation.h"

using namespace std;

const char* fileName = "c_hdf5_plt_cnt_1000.txt";
//const char* fileName = "first(phillip's).txt";

const char* outFileName = "logN-vs-logE.txt";


void printArray(int** arrayIn, int HEIGHT, int WIDTH){
	cout << endl;
	for(int i = 0; i < HEIGHT; i++){
		for(int j = 0; j < WIDTH; j++){
			printf("%3d", arrayIn[i][j]);
		}
		printf("\n");
	 }
}

void printArray(double** arrayIn, double HEIGHT, double WIDTH){
	cout << endl;
	printf("%7s%7s\n", "level", "log(n)");
 	for(int i = 0; i < HEIGHT; i++){
		for(int j = 0; j < WIDTH; j++){
			printf("%7.3f ", arrayIn[i][j]);
		}
		printf("\n");
 	}
}

void printToFile(double** arrayIn, int HEIGHT, int WIDTH, double slope, double yint){
	FILE * pFile;
	pFile = fopen(outFileName, "w");
	fprintf(pFile, "slope: %f\nintercept: %f\n", slope, yint);
	for(int i = HEIGHT - 1; i >= 0; i--){
		for(int j = 0; j < WIDTH; j++){
			fprintf(pFile, "%-2.6f ", arrayIn[i][j]);
		}
		fprintf(pFile, "\n");
	}
	fclose(pFile);
}

/**
	Returns the log base 2 of the number of cells filled in a given level
	of fractal analysis.
	@param arrayIn is the 2-3-D array of integers that contains the data
	@param HEIGHT
	@param WIDTH
	@param DPETH
	@param level is the level of analysis to be performed. Level 0, for example, 
	examines only the individual data points, whereas at level 1 they are merged
	into boxes of side length 2^1, and for level l size 2^l.
	@return log base 2 of the number of cells filled in a given level.

**/
double boxCounting(int*** arrayIn, int HEIGHT, int WIDTH, int DEPTH, int level){
	int numberFilled = 0; //number of boxes with an element of the figure
	int boxDimension = (int) pow(2, level);
	int boxDimensionZ;

	if(DEPTH != 1){
		boxDimensionZ = boxDimension;
	}
	else{
		boxDimensionZ = 1;
	}
	//iterates through all boxes
	//cout << "here!" << endl;
	for(int i = 0; i < HEIGHT; i += boxDimension){
		for(int j = 0; j < WIDTH; j += boxDimension){
			for(int k = 0; k < DEPTH; k += boxDimensionZ){
				int boxSum = 0;


				//sums each box
				for(int boxSumX = 0; boxSumX < boxDimension; boxSumX++){
					for(int boxSumY = 0; boxSumY < boxDimension; boxSumY++){
						for(int boxSumZ = 0; boxSumZ < boxDimensionZ; boxSumZ++){
							//if(boxSumX + i >= HEIGHT || boxSumY + j >= WIDTH || boxSumZ + k >= DEPTH){ This is to deal with dimensions that aren't
							//	cout << endl << "Dimensions must be powers of two!" << endl;				powers of two.
							//	boxSum += 0;
							//}
							//else{
								boxSum += arrayIn[boxSumX + i][boxSumY + j][boxSumZ + k];
							//}
						}
					}
				}

				if(boxSum > 0){
					numberFilled++;
				}

			}
		}
	}
	printf("%s%d\n%s%d\n%s%d\n\n", "level: ", level, 
	"--box dimension: ", boxDimension, "--filled boxes: ", numberFilled);		

	return log2(numberFilled);
}

int main () {
	int HEIGHT, WIDTH, DEPTH;
	double*** elements;
	bool haveZeros = false;
	double arraySum;
	int*** elementsInterpolated;

	elements = dataReaderASCII<double>(fileName, HEIGHT, WIDTH, DEPTH, haveZeros, arraySum); //change to dataReaderASCII to read in text files
	elementsInterpolated = interpolate(elements, HEIGHT, WIDTH, DEPTH, 0.5);
	//printToFile(elementsInterpolated, HEIGHT, WIDTH);

	int LOWESTLEVEL = 0;

	//initialize out-array
	int largestDimension = fmin(HEIGHT, WIDTH);
	if(DEPTH > 1){
		largestDimension = fmin(largestDimension, DEPTH);
	}
	int outArrayLength = int(log2(largestDimension) - LOWESTLEVEL + 1);
	double** outDataArray = new double* [outArrayLength];

	//printf("\n\n%d\n\n", outArrayLength);

	for(int i = 0; i < outArrayLength; i++){
		outDataArray[i] = new double[2];
	}

	for(int k = 0; k < outArrayLength; k++)
	{
		outDataArray[k][0] = outArrayLength - 1 - (k + LOWESTLEVEL);
		outDataArray[k][1] = boxCounting(elementsInterpolated, HEIGHT, WIDTH, DEPTH, k + LOWESTLEVEL);
	}

	printArray(outDataArray, outArrayLength, 2);

	double* regressionArray = new double[3];

	slope(outDataArray, outArrayLength, regressionArray);

	printf("\n\n%s%s\n\n%s%.5f\n%s%1.3f\n\n", "Curve: ", fileName, "Dimension: ", regressionArray[0], "R^2: ", regressionArray[1]);

	printToFile(outDataArray, outArrayLength, 2, regressionArray[0], regressionArray[2]);

	delete[] elements;
	delete[] elementsInterpolated;
	delete[] outDataArray;
	delete[] regressionArray;

	return 0;
}

/**
	Module to interpolate the isocontour at some pre-specified value inside a 2- or 3-D array.

	Implemented in boxCounter3D.cpp:
		Fractal analysis module utilizing boxcounting to determine 
		the fractal dimension of a shape in the input text/binary file.

		Prints to terminal:
			- the results at each level of analysis
			- the overall dimension as determined by least-squares regression
			- an R^2 value
			
		@author Samuel Brenner
		@version July 11, 2013

	@author Samuel Brenner
	@version July 11, 2013

**/

#ifndef interpolation_h
#define interpolation_h

int*** interpolate(double*** arrayIn, int HEIGHT, int WIDTH, int DEPTH, double valueToFind){
	//Declare and initialize arrayOut. If any values in the arrayIn happen to be the valueToFind,
	//we'll initialize the corresponding arrayOut to a 1; otherwise it'll be a 0.
	int*** arrayOut = new int**[HEIGHT];
	for(int i = 0; i < HEIGHT; i++){
		arrayOut[i] = new int*[WIDTH];
		for(int j = 0; j < WIDTH; j++){
			arrayOut[i][j] = new int[DEPTH];
			for(int k = 0; k < DEPTH; k++){
				//performs a preliminary check so that we don't miss any values that are exactly == valueToFind
				if(arrayIn[i][j][k] == valueToFind){
					arrayOut[i][j][k] = 1;
				}
				else{
					arrayOut[i][j][k] = 0;
				}
			}
		}
	}

	//Makes the interpolater able to handle planar data
	int startingDepthIndex = 0;
	if(DEPTH != 1){
		startingDepthIndex = 1;
	}
	else{
		DEPTH++;
	}

	/*
		Iterates over all entries in the array that aren't on the edge of the array, unless the array is 2D: for 2D
		arrays the program ignores the edge condition in the third dimension.

		For each entry analyzed, we check to see if the value is lower than valueToFind. If so, the neighbors on all 
		six sides (four if planar data) are checked against it to determine if there is some crossing over valueToFind
		in between the two. Then we determine which cell should contain that crossing and assign it a 1 in the 
		corresponding cell of the outArray.
	*/
	for(int i = 1; i < HEIGHT - 1; i++){
		for(int j = 1; j < WIDTH -1; j++){
			for(int k = startingDepthIndex; k < DEPTH - 1; k++){
				if(arrayIn[i][j][k] < valueToFind){
					if(arrayIn[i + 1][j][k] > valueToFind){
						if(int((valueToFind - arrayIn[i][j][k]) / (arrayIn[i + 1][j][k] - arrayIn[i][j][k])) == 0){
							arrayOut[i][j][k] = 1;
						}
						else{
							arrayOut[i + 1][j][k] = 1;
						}
					}
					if(arrayIn[i][j + 1][k] > valueToFind){
						if(int((valueToFind - arrayIn[i][j][k]) / (arrayIn[i][j + 1][k] - arrayIn[i][j][k])) == 0){
							arrayOut[i][j][k] = 1;
						}
						else{
							arrayOut[i][j + 1][k] = 1;
						}
					}
					if(arrayIn[i][j - 1][k] > valueToFind){
						if(int((valueToFind - arrayIn[i][j][k]) / (arrayIn[i][j - 1][k] - arrayIn[i][j][k])) == 0){
							arrayOut[i][j][k] = 1;
						}
						else{
							arrayOut[i][j - 1][k] = 1;
						}
					}
					if(arrayIn[i - 1][j][k] > valueToFind){
						if(int((valueToFind - arrayIn[i][j][k]) / (arrayIn[i - 1][j][k] - arrayIn[i][j][k])) == 0){
							arrayOut[i][j][k] = 1;
						}
						else{
							arrayOut[i - 1][j][k] = 1;
						}
					}
					if(arrayIn[i][j][k + 1] > valueToFind){
						if(int((valueToFind - arrayIn[i][j][k]) / (arrayIn[i][j][k + 1] - arrayIn[i][j][k])) == 0){
							arrayOut[i][j][k] = 1;
						}
						else{
							arrayOut[i][j][k + 1] = 1;
						}
					}
					if(arrayIn[i][j][k - 1] > valueToFind){
						if(int((valueToFind - arrayIn[i][j][k]) / (arrayIn[i][j][k - 1] - arrayIn[i][j][k])) == 0){
							arrayOut[i][j][k] = 1;
						}
						else{
							arrayOut[i][j][k - 1] = 1;
						}
					}

				}
			}
		}
	}

return arrayOut;
}

#endif

/**
	Module to run regression for boxcounting.
	
	Changes a double[3] such that regressionArray[0] = slope,
	regressionArray[1] = R^2, and regressionArray[2] = y-int.

	Implemented in boxCounter3D.cpp:
		Fractal analysis module utilizing boxcounting to determine 
		the fractal dimension of a shape in the input text/binary file.

		Prints to terminal:
			- the results at each level of analysis
			- the overall dimension as determined by least-squares regression
			- an R^2 value
			
		@author Samuel Brenner
		@version July 11, 2013

	@author Samuel Brenner
	@version July 11, 2013

**/


#include "regressionModule.h"

#include <math.h>

double stdev(double* arrayIn, double arrayMean, int arrayLength);

double corrCoeff(double* arrayX, double* arrayY, double meanX, 
	double meanY, double stdevX, double stdevY, int length);

double mean(double* arrayIn, int arrayLength);

void slope(double** arrayIn, int arrayLength, double* regressionArray){
	
	double* arrayX = new double[3];

	for(int i = 0; i < 3; i++){
		arrayX[i] = arrayIn[i][0];
	}

	double* arrayY = new double[3];
	for(int i = 0; i < 3; i++){
		arrayY[i] = arrayIn[i][1];
	}

	double meanX = mean(arrayX, 3);
	double meanY = mean(arrayY, 3);

	double stdevX = stdev(arrayX, meanX, 3);
	double stdevY = stdev(arrayY, meanY, 3);

	double r = corrCoeff(arrayX, arrayY, meanX, meanY, stdevX, stdevY, 3);

	double slope = r 
		* stdevY / stdevX;


	delete[] arrayX;
 	delete[] arrayY;

 	regressionArray[0] = slope;
 	regressionArray[1] = pow(r, 2);
 	regressionArray[2] = meanY - slope * meanX;
}

double stdev(double* arrayIn, double arrayMean, int arrayLength){

	double squaresSum = 0;

	for(int i = 0; i < arrayLength; i++){
		squaresSum += pow(arrayIn[i] - arrayMean, 2);
	}

	return sqrt(squaresSum / (arrayLength - 1));
}

double corrCoeff(double* arrayX, double* arrayY, double meanX, 
	double meanY, double stdevX, double stdevY, int length){

	double sum = 0;

	for(int i = 0; i < length; i++){
		sum += (arrayX[i] - meanX) * (arrayY[i] - meanY);
	}

	return sum / (stdevX * stdevY * (length - 1));
}

double mean(double* arrayIn, int arrayLength){
	double sum = 0;

	for(int i = 0; i < arrayLength; i++){
		sum += arrayIn[i];
	}

	return sum / arrayLength;
}

\end{lstlisting}