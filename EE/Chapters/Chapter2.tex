% Chapter Template

\chapter{A computational approach to box-counting dimension} % Main chapter title

\label{Chapter2} % Change X to a consecutive number; for referencing this chapter elsewhere, use \ref{ChapterX}

\lhead{Chapter 2. \emph{Computational box-counting}} % Change X to a consecutive number; this is for the header on each page - perhaps a shortened title

\begin{myexample} We can apply numerical techniques to verify the agreement of this definition with that established in Section \ref{intuitivedimension}. A computational module to calculate the box-counting dimension is included in \ref{AppendixA} and is here applied to a Koch curve of size ???? pixels end-to-end. 

****include graphics of curve and results, compare dimensions and error****  \end{myexample}

\begin{myexample}Same thing, Sierpinski pyramid, perhaps?\end{myexample}


\begin{table*}
\begin{center}
\caption{Theoretical vs. Calculated Fractal Dimensions Found in Fractal Analysis Validation}\label{t:table}
\begin{tabular}{lcccc}
Fractal Object 				&	Theoretical	&	Calculated 	&	Error		&	\% error\\
\hline\\
Dragon boundary				&	1.5236					&	1.487	                &	-0.037		&	2.4	\\
Vicsek Fractal (Box)		& 	1.4649					&	1.3264					&	-0.14		&	9.5	\\
Fibonacci Word				&	1.6379					&	1.45964					&	-0.18 		&	11	\\
Gosper Curve				&	2						&	1.72811					&	-0.27		&	14				\\
Boundary of Gosper Island	&	1.1292					&	1.2056					&	0.077		&	6.8	\\
Julia Set					&	2						&	1.49408					&	-0.51		&	25	\\
Julia $z^2+1/4$				&	1.0812					&	1.19677					&	0.12		&	11	\\
Levy C boundary				&	1.934					&	1.5656					&	-0.37		&	19	\\
Pythagoras Tree				&	2						&	1.74588					&	-0.25		&	13	\\
\hline
\rule{0pt}{4ex}				&							&							& \textbf{Average: } & 12\\
\end{tabular}
\end{center}
\end{table*}
