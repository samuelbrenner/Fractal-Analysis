% APJ Reference: http://aas.org/journals/authors/common_instruct#_Toc2

\documentclass[iop]{emulateapj}
% \documentclass[12pt,preprint]{aastex}


% Define the sun symbol for use with solar mass
% \newcommand{\Sun}{\odot}
\newcommand{\vect}[1]{\boldsymbol{#1}}

% \usepackage{amssymb}
%\usepackage{xfrac}
% \usepackage{amsmath}
\usepackage{hyperref}
% \usepackage{footnote}
% \usepackage{multirow}
% \usepackage{array}
% \usepackage[lofdepth,lotdepth]{subfig}
% \usepackage{todonotes}
% \usepackage{float}

\hypersetup{
    colorlinks,
    citecolor=blue,
    filecolor=black,
    linkcolor=blue,
    urlcolor=black
}

% \newcolumntype{x}[1]{%
% >{\centering\hspace{0pt}}p{#1}}%

\newcommand{\tn}{\tabularnewline}


\bibliographystyle{apj}

\shortauthors{Handy, Plewa, Odrzwolek}
\shorttitle{Characterizing stellar phenomena with fractals and multifractals}

%--------------------
%	Begin document
%--------------------
\begin{document}
%
\title{Fractal and Multifractal Analysis as Tools to Characterize Supernovae and Molecular Clouds: \\
subtitle}
%
\author{Samuel Brenner\altaffilmark{1}}
%
%\author{Tomasz Plewa\altaffilmark{1}}
%\email{tplewa@fsu.edu}
%
%\author{Andrzej Odrzwolek\altaffilmark{2}}
%\email{andrzej's email}
%
\altaffiltext{1}{Young Scholars Program, Florida State University \email{samuel.e.brenner@gmail.com}}
%
%
%
%--------------------
%	Begin Abstract
%--------------------
\begin{abstract}
\end{abstract}
%
%
%
%--------------------
%	Begin keywords
%--------------------
\keywords{supernovae, fractals, multifractals}
%
%
%
%--------------------
%	Begin Body
%--------------------

\section{Introduction}
Many physical phenomena cannot be characterized by the idealizations of Euclidean geometry alone; they exhibit ``roughness''. That is, they have a detailed structure at any arbitrarily small size scale \citep{fractaltextbook}. The development of \textit{fractal geometry} allows a mathematical treatment of the ``roughness'' inherent in the non-idealized phenomena of the real world. Multifractal analysis permits us to examine how those fractal characteristics themselves change with scale; in fact, they may even be fractal themselves. In this paper, we detail the application of fractal geometry to flame fronts in stars\footnote{there's gotta be a better way to say this} and then analyze the multifractal characteristics of the cool molecular clouds involved in star formation.

\subsection{Fractal analysis of type 1a supernova flame fronts}
The accepted model for a type 1a supernova is a white dwarf that accumulates mass from a binary companion until it reaches a mass so large that the compression of the heavier elements in the core causes them to ignite and fuse, releasing more energy in the process. This deflagration results in a flame front that spreads rapidly throughout the star and causes the visible effects of supernova\footnote{it does, right?}.

The surface of the expanding flame front can be dramatically affected by multiple types of turbulence. \cite{Blinnikov1996} showed that 

\subsection{Mutlfractals}

\subsection{Physical phenomena studied}

\section{Methods}
World

\section{Results}
Diving pretty deep here

\section{Analysis}
Test

\section{Discussion}



%
%
%
%--------------------
%	Begin ackownledgements
%--------------------
\section{Acknowledgments}\label{s:ack}
%
The work of SB has been supported by Thomasz Plewa and Tim Handy at Florida State University under the Young Scholars Program.
%
%
%
%--------------------
%	Begin references
%--------------------
\bibliographystyle{apj}
\bibliography{main}
%
%
%
\end{document}
