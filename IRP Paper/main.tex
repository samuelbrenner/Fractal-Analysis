% APJ Reference: http://aas.org/journals/authors/common_instruct#_Toc2

\documentclass[iop]{emulateapj}
% \documentclass[12pt,preprint]{aastex}


% Define the sun symbol for use with solar mass
% \newcommand{\Sun}{\odot}
\newcommand{\vect}[1]{\boldsymbol{#1}}

% \usepackage{amssymb}
%\usepackage{xfrac}
% \usepackage{amsmath}
\usepackage{hyperref}
% \usepackage{multirow}
% \usepackage{array}
% \usepackage[lofdepth,lotdepth]{subfig}
% \usepackage{todonotes}
% \usepackage{float}

\hypersetup{
    colorlinks,
    citecolor=blue,
    filecolor=black,
    linkcolor=blue,
    urlcolor=black
}

% \newcolumntype{x}[1]{%
% >{\centering\hspace{0pt}}p{#1}}%

\newcommand{\tn}{\tabularnewline}


\bibliographystyle{apj}

\shortauthors{Handy, Plewa, Odrzwolek}
\shorttitle{Characterizing stellar phenomena with fractals and multifractals}

%--------------------
%	Begin document
%--------------------
\begin{document}
%
\title{Fractal and Multifractal Analysis as Tools to Characterize Supernovae and Molecular Clouds: \\
subtitle}
%
\author{Samuel Brenner\altaffilmark{1}}
\email{samuel.e.brenner@gmail.com}
%
%\author{Tomasz Plewa\altaffilmark{1}}
%\email{tplewa@fsu.edu}
%
%\author{Andrzej Odrzwolek\altaffilmark{2}}
%\email{andrzej's email}
%
\altaffiltext{1}{Young Scholars Program, Florida State University}
% \altaffiltext{2}{Jagiellonian University}
%
%
%
%--------------------
%	Begin Abstract
%--------------------
\begin{abstract}
\textit{so far this is just the introduction...} Many physical phenomena cannot be characterized by Euclidean geometry alone. The development of \textit{fractal geometry} allows a mathematical treatment of the ``roughness'' inherent in the non-idealized phenomena of the real world. In this paper, we detail the application of fractal geometry to various boundaries in stars \textbf{there's gotta be a better way to say this} and then move to analyze the multifractal characteristics of ...
\end{abstract}
%
%
%
%--------------------
%	Begin keywords
%--------------------
\keywords{supernovae}
%
%
%
%--------------------
%	Begin Body
%--------------------

\section{Introduction}
Many physical phenomena cannot be characterized by the idealizations of Euclidean geometry alone; they exhibit ``roughness'' that  . The development of \textit{fractal geometry} allows a mathematical treatment of the ``roughness'' inherent in the non-idealized phenomena of the real world. In this paper, we detail the application of fractal geometry to various boundaries in stars \textbf{there's gotta be a better way to say this} and then move to analyze the multifractal characteristics of 

\section{Methods}
World

\section{Results}
Diving pretty deep here

\section{Analysis}
Test

\section{Discussion}


\cite{kifonidis+03}

\citep{kifonidis+06}

\citet{kifonidis+06}

\cite{kifonidis+03,kifonidis+06}

\citep{kifonidis+03,kifonidis+06}

\citet{kifonidis+03,kifonidis+06}

%
%
%
%--------------------
%	Begin ackownledgements
%--------------------
\section{Acknowledgments}\label{s:ack}
%
The work of SB has been supported by Thomasz Plewa and Tim Handy at Florida State University under the Young Scholars Program.
%
%
%
%--------------------
%	Begin references
%--------------------
\bibliographystyle{apj}
\bibliography{main}
%
%
%
\end{document}
